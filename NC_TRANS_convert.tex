\documentclass{article}

\begin{document}

The following is how NITROGEN calculates the NC\_TRANS matrix and how CFOUR can be used to generate the same.

First, we need the F matirx (Hessian). CFOUR's units are in Bohr while NITROGEN uses Angstrom. Therefore, every internal coordinate from FCMINT needs to be scaled accordingly.
Angles should remain untouched, so each row with a radius needs to be converted as well as each column with a radius.
I am unsure why, but for the water molecule, row three and column three need to be converted to angstroms in the G matrix.

$L^{-1}GFL = \Lambda$

$L$ and $\Lambda$ are column swapped so that the eigenvalues are sorted in ascending order.
Then comes a weird chunk of code I found in surf.c (NITROGEN) that scales $L$ so that $L^{-1}FL$ yields the diagonalized $F$ with frequencies along the diagonal.
See 'NITROGEN' for those notes.
Then $L^{-1 T}$ yields the left block of NC\_TRANS. The right most column is
$L^{-1}q$
Where $q$ is the IC eq. geometry.


\end{document}
